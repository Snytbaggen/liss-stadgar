\documentclass[a4paper]{article}
\usepackage[utf8]{inputenc}
\usepackage[T1]{fontenc}
\usepackage[margin=2.8cm]{geometry}

\renewcommand{\thesection}{§ \arabic{section}}
\renewcommand{\thesubsection}{\arabic{section}.\arabic{subsection}}

\title{Linköpings StudentSpex stadgar}
\author{}
\date{15 oktober 2018}

\begin{document}
\maketitle
%\tableofcontents

\section{Syfte}
Linköpings StudentSpex är en ideell förening med säte i Linköping., och har som syfte att att hålla spextraditionen i Linköping vid liv. Detta görs genom att bland annat sätta upp egna spexföreställningar, bjuda in andra spex till Linköping samt anordna resor för att besöka andra spexföreningar.

\section{Medlemskap}
Medlemmar ska främst rekryteras bland studenter vid Linköpings Universitet. Man är medlem om man ansluter sig till föreningens syfte, betalar medlemsavgiften för nuvarande verksamhetsår samt arbetar i en spexdirections organisation eller tidigare har varit medlem i föreningen.\newline
\newline
Livstidsmedlemskap kan lösas efter tre år som betalande medlem, förutsatt att övriga kriterier för medlemskap är uppfyllda. Livstidsmedlemskap löses genom att medlemmen väljer att betala livstidsmedlemskapsavgiften istället för den ordinarie medlemsavgiften. Livstidsmedlemskap medför samma rättigheter och skyldigheter som ordinarie medlemskap, med skillnaden att medlemskapet förnyas för varje nytt verksamhetsår utan att någon ny avgift behöver betalas. Avgiften för livstidsmedlemskap fastställs av årsmötet. En livstidsmedlem kan när som helst begära utträde ur föreningen genom att skriftligen meddela styrelsen, och ska då behandlas som en ordinarie medlem vars medlemskap förfallit på grund av obetald medlemsavgift för nuvarande verksamhetsår.\newline
\newline
Hedersmedlemskap kan enbart ges av föreningsmötet. En hedersmedlem har samma rättigheter som en ordinarie medlem, men inte några av dess skyldigheter.

\section{Verksamhetsår}

Verksamhetsåret är brutet, och löper mellan 1 juli och 30 juni.

\section{Organisation}
Föreningens organisation utövas genom:

\begin{enumerate}
  \item Föreningsmöten (\ref{section:föreningsmötet})
  \item Styrelsen (\ref{section:styrelsen})
  \item Spexdirectionen \ref{section:spexdirectionen})
  \item Synopsisgruppen (\ref{section:synopsisgruppen})
  \item Valberedningen (\ref{section:valberedningen})
  \item Revisorerna (\ref{section:revisorerna})
\end{enumerate}

\section{Föreningsmötet}
\label{section:föreningsmötet}
Föreningsmötet är föreningens högsta beslutande organ. Alla medlemmar har närvaro-, yttrande-, förslags- och rösträtt. Föreningsmötet har rätt att adjungera in personer med närvaro-, yttrande- och förslagsrätt. Föreningsmötet ska ha minst två ordinarie sammanträdanden per verksamhetshalvår. Verksamhetsårets första möte, "höstmötet", ska hållas innan den 15 oktober. "Årsmötet" ska hållas under verksamhetsårets andra halva, mellan 15 april och 31 maj.

Föreningsmötet kallas av styrelsen. Extrainsatta föreningsmöten kan begäras av revisorer, spexdirecteuren eller en grupp om minst tio medlemmar. Föreningsmötet är beslutsmässigt om minst tolv medlemmar är närvarande. Föreningsmötet får ej ligga under sommar-, jul- och påskledigheter. Protokoll från föreningsmötet ska finnas lätt tillgängligt för föreningens medlemmar senast tre veckor efter mötet.

Preliminär kallelse och kallelse skall skriftligen utsändas 14 respektive 7 dagar före ordinarie eller extrainsatt föreningsmöte. Kallelsen skall innehålla en föredragningslista samt en förteckning över möteshandlingar, vilka även ska finnas tillgängliga under föreningsmötet.

Föreningsmötet har rätt att entlediga en förtroendevald medlem från dess post. Fyllnadsval förrättas av föreningsmötet.

Vid lika röster efter votering ska lotten avgöra vilket beslut som fattas.\newline

\noindent
\newline
Föreningmötet skall
\begin{itemize}
  \item välja styrelse, revisorer och valberedning
  \item fastställa en verksamhetsplan för styrelsen
  \item välja spexdirection
  \item välja synopsisgrupp
  \item fastställa budget för föreningens verksamhet
  \item fastställa budget för spexdirectionens verksamhet
\end{itemize}

\noindent
\newline
Vid föreningsmöte ska följande punkter tas upp:
\begin{enumerate}
  \item Val av mötesordförande
  \item Val av mötessekreterare
  \item Val av två justeringsmän, tillika rösträknare
  \item Mötets stadgeenliga utlysande
  \item Eventuella adjungeringar
  \item Fastställande av föredragningslista
  \item Föregående föreningsmötes protokoll
  \item Rapporter och meddelanden
  \item Övriga frågor
\end{enumerate}

\noindent
\newline
På höstmötet ska även följande punkter tas upp:
\begin{itemize}
  \item Fråga om ansvarsfrihet för föregående verksamhetsårs styrelse
  \item Fastställande av budget för innevarande verksamhetsår
  \item Val av valberedning
\end{itemize}

\noindent
\newline
På årsmötet ska även följande punkter tas upp:
\begin{itemize}
  \item Val av styrelse för kommande verksamhetsår
  \item Val av två revisorer för kommande verksamhetsår
  \item Fastställande av medlemsavgifter för kommande verksamhetsår
\end{itemize}

\section{Styrelsen}
\label{section:styrelsen}
Styrelsen utgör den omedelbara ledningen för föreningen och företräder föreningen mellan föreningsmöten.\newline
\newline
Styrelsen består av ordförande, vice ordförande, sekreterare, kassör, förvaltare samt IT-ansvarig.\newline
\newline
Styrelsen väljs av årsmötet för ett verksamhetsår. Samtliga ledamöter skall vara myndiga.\newline
\newline
Styrelsen sammanträder på kallelse av ordförande. Kallelse skall vara ledamöterna tillhanda minst fyra dagar före sammanträdet. Till varje styrelsemöte skall också aktiva revisorer och spexdirectioner kallas enligt ovanstående regler och automatiskt adjungeras.\newline
\newline
Styrelsen är beslutsmässig om minst fyra styrelseledamöter är närvarande, därav ordföranden eller vice ordföranden.\newline
\newline
Ordföranden och kassören tecknar var för sig föreningens firma.\newline
\newline
\textbf{Styrelsen} åligger

\begin{itemize}
  \item att kalla medlemmarna till föreningsmöten
  \item att bereda ärenden som skall behandlas av föreningsmötet
  \item att inför föreningsmötet ansvara för föreningens ekonomi, bortsett från spexdirectionens verksamhet
  \item att verkställa av föreningsmötet fattade beslut
  \item att handha föreningens administrativa göromål
  \item att för höstmötet framlägga budgetförslag för föreningens verksamhet
  \item att noggrant följa upp av föreningsmötet antagen budget
  \item att vid avslutad verksamhetsperiod avge berättelse över den företagna verksamheten
  \item att den 28 november varje år arrangera föreningens födelsedagskalas
  \item att underlätta medlemmarnas kontakt med andra spex och därtill hörande arrangemang, t.ex. genom anordnande av resor
\end{itemize}

\noindent
\textbf{Ordföranden} åligger

\begin{itemize}
  \item att representera föreningen och föra dess talan
  \item att leda och övervaka arbetet inom styrelsen
  \item att kalla till och leda styrelsens sammanträden
  \item att hålla kontakt med föreningens övriga funktionärer
  \item att hålla föreningens medlemmar väl informerade om föreningens verksamhet
\end{itemize}

\noindent
\textbf{Vice ordföranden} åligger

\begin{itemize}
  \item att vara ordföranden behjälplig i dennes sysslor
  \item att leda styrelsesammanträdena då ordföranden är frånvarande
  \item att vidmakthålla och utveckla kontakten med andra spex
  \item att vara kontaktperson för gästande spex
\end{itemize}

\noindent
\textbf{Kassören} åligger

\begin{itemize}
  \item att handha föreningens kassa
  \item att framta föreningens budget
  \item att löpande hålla styrelsen och dess revisorer informerade om det ekonomiska läget
  \item att se till att ekonomi och redovisning sköts löpande och korrekt för föreningen inklusive aktiva spexuppsättningar
  \item att upprätta föreningen bokslut
  \item att se till att föreningens bokslut är revisorerna tillhanda senast tre veckor innan höstmötet efter uppdragstidens utgång
\end{itemize}

\noindent
\textbf{Sekreteraren} åligger

\begin{itemize}
  \item att föra styrelsens protokoll
  \item att tillse att alla medlemmar har tillgång till aktuella stadgar
\end{itemize}

\noindent
\textbf{Förvaltaren} åligger

\begin{itemize}
  \item att förvalta föreningens prylar, såsom lokaler, reliker och arkivalier
  \item att hålla föreningens inventarieförteckning
  \item att administrera inköp och försäljning av tröjor, dekaler, overaller och liknande
\end{itemize}

\noindent
\textbf{Den IT-ansvarige} åligger

\begin{itemize}
  \item att ansvara för driften av föreningens IT-tjänster
  \item att sköta föreingens medlemsmatrikel, inklusive 100-klubben
  \item att administrera föreingens externa kommunikationskanaler
\end{itemize}

\noindent
Vid lika röstetal efter votering i någon fråga, gäller den mening som sammanträdets ordförande företräder.\newline
\newline
Om beslut om ansvarsfrihet för föregående verksamhetsårs styrelse ej kan fattas på höstmötet åligger det styrelsen i fråga att avge en statusrapport för verksamhetsårets avslutande på varje påföljande föreningsmöte till dess beslut om ansvarsfrihet kan fattas.\newline

\section{Spexdirectionen}
\label{section:spexdirectionen}
Spexdirectionen utgör ledningen för en spexuppsättning.\newline
\newline
Spexdirectionen består av en spexdirecteur och en ekonomichef.\newline
\newline
Om speciella omständigheter och behov så påkallar kan föreningsmötet besluta att spexdirectionen för en viss spexuppsättning skall ha en annan sammansättning.\newline
\newline
En spexdirection väljs då föreningsmötet finner att tiden är inne för uppstartande av en nytt spexuppsättning. Spexdirectionen är ansvarig inför föreningsmötet. Under en övergångsperiod kan flera spexdirectioner verka.\newline
\newline
Organisationen ska kallas SPEX-nn, där nn är årtalet för spexuppsättningens premiär. Om omständigheterna kräver kan annan unik benämning för uppsättningen väljas.\newline
\newline
Spexdirecteur och ekonomichef äger rätt att var för sig teckna föreningens firma i organisationens namn.\newline
\newline
\textbf{Spexdirectionen} åligger

\begin{itemize}
  \item att inför föreningsmötet framlägga budget för sin verksamhet
  \item att hålla föreningens medlemmar väl informerade om sin verksamhet
  \item att ansvara för spexdirectionens organisations ekonomi
\end{itemize}

\noindent
\textbf{Spexdirecteuren} åligger

\begin{itemize}
  \item att på föreningsmötets uppdrag genomföra en spexuppsättning enligt gängse tradition
\end{itemize}

\noindent
\textbf{Ekonomichefen} åligger:

\begin{itemize}
  \item att i samråd med kassören se till att ekonomi och redovisning sköts löpande och korrekt
  \item att vara spexdirecteuren behjälplig i dennes sysslor
  \item att ständigt hålla spexdirecteuren och spexdirectionens revisorer informerade om det ekonomiska läget
  \item att noggrant följa upp av föreningsmötet antagen budget
\end{itemize}

\noindent
Samtliga i organisationen arbetande personer måste vara medlemmar i föreningen. Spexdirectionen kan under kortare tid engagera personer som inte är föreningsmedlemmar till arbetet med uppsättningen.\newline
\newline
Då spexdirectionens uppsättning anses nedlagd ska verksamheten snarast revideras. Spexdirectionen befrias av föreningsmötet från sina uppgifter då dennas spexrevision är godkänd.\newline
\newline
Spexuppsättningens räkenskaper och övriga handlingar ska vara revisorerna tillhanda senast tre veckor före det föreningsmöte som är det närmast påföljande fyra månader efter spexuppsättningens sista föreställning.\newline
\newline
Om SPEX-nn inte kan avslutas enligt denna stadga så åligger det spexdirectionen att avge en statusrapport för spexuppsättningens avslutande på varje påföljande föreningsmöte till dess SPEX-nn kan avslutas.

\section{Synopsisgruppen}
\label{section:synopsisgruppen}

Synopsisgruppen väljs av ett föreningsmöte.\newline
\newline
Synopsisgruppen åligger

\begin{itemize}
  \item att organisera och utlysa synopsistävling samt sammanställa resultatet
  \item att hemlighålla samtliga synopsisförslag för andra än föreningens medlemmar 
  \item att överlämna samtliga inkomna förslag till förvaltaren för arkivering
\end{itemize}

\noindent
Då den nye spexdirecteuren är vald tar denne över ansvaret för synopsisgruppens arbete.

\section{Valberedningen}
\label{section:valberedningen}

Valberedningen består av minst tre personer och väljs av höstmötet.\newline
\newline
Valberedningen åligger att framta förslag på kandidater till de poster som föreningsmötet har att välja.

\section{Revisorerna}
\label{section:revisorerna}

Av föreningsmötet valda funktionärers arbete skall granskas av två revisorer. Revisor skall vara myndig och får ej inneha post som han/hon är utsedd att granska.\newline
\newline
Revisorerna väljs vid det föreningsmöte som väljer den spexdirektion, styrelse eller annan spexverksamhet som de ska revidera. En revisor kan vara ålagd att granska flera spexverksamheter.\newline
\newline
Revisorerna åligger

\begin{itemize}
  \item att revidera granskad spexverksamhets räkenskaper
  \item att avge berättelse för företagna revisioner
  \item att kontinuerligt övervaka att granskad spexverksamhet följer föreningen stadgar och intentioner
  \item att se till att valda funktionärer får de råd och det stöd som behövs för utföra ålagda uppgifter
  \item att minst var sjätte månad, eller närhelst det anses befogat, inhämta en aktuell resultat- och balansrapport för granskad spexverksamhet och föredra en statusrapport på nästkommande föreningsmöte
\end{itemize}

\noindent
Revisionsberättelse skall även innehålla yttrande i fråga om ansvarsfrihet för berörda organ och personer.\newline
\newline
Räkenskaper samt övriga handlingar skall om revisorerna inte medger annat tillställas revisorerna senast tre veckor före det föreningsmöte vid vilket revisionsberättelse eller annan rapport skall avges.\newline
\newline
Avgår någon av föreningsmötet vald funktionär under verksamhetsperioden skall hans/hennes verksamhet omedelbart revideras.

\section{100-klubben}
\label{section:100-klubben}

Föreningsmedlem som varit närvarande vid 100 stycken av föreningens föreställningar inför betalande publik blir automatiskt medlem i 100-klubben.\newline
\newline
Föreningsmedlem som inte är aktiv i ett års spexuppsättning eller styrelse, kan max tillgodoräkna sig två föreställningar detta år.\newline
\newline
Medlem i 100-klubben äger rätt att erhålla ett medlemsbevis. Medlemsbevisen bekostas av föreningen.\newline
\newline
Medlemskapet innebär inga kostnader för medlem.

\section{Stadgetolkning}
Skulle tvist uppstå angående tolkning av dessa stadgar hänskjutes frågan till Linköpings Universitets Studentkårers presidium för avgörande.

\section{Stadgeändring}
För ändring i denna stadga krävs 2/3 majoritet för ändringsförslaget på två efter varandra följande föreningsmöten med minst 30 dagars mellanrum.

\section{Spexets upphörande}
Föreningen skall upphöra om antalet ordinarie medlemmar understiger tolv, styrelsen oräknad, eller i det fall årsmötet med 3/4 majoritet så beslutar. Föreningens tillgångar fonderas i så fall hos Linköpings Universitets Studentkårer. Fonden utbetalas till nybildad studentorganisation med spexliknande verksamhet. Föreningen kan dock ej upphöra om skulderna överstiger tillgångarna. Föreningsmötes beslut om spexets upphörande måste ske under punkt upptagen i kallelse.

\setcounter{section}{16}
\section{ }
\textbf{Det fixar sig.}

\end{document}
